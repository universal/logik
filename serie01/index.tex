\documentclass[12pt,a4paper]{scrartcl}
\usepackage{ifpdf}
\usepackage[utf8]{inputenc}
\usepackage[ngerman]{babel}
\usepackage{amsmath}
\usepackage{amssymb}
\title{}
\author{Johannes Hedtrich, Christopher Hlubek}
\date{\today}
\begin{document}
Johannes Hedtrich, Christopher Hlubek\\
\section*{1.4 Induktive Definition}
Sei $d$ eine Funktion $d: F_{AL} \rightarrow \mathbb{N}$, die zu jeder aussagenlogischen Formel $\psi \in F_{AL}$ die Länge von $\psi$ liefert.

\noindent
\textbf{1. Basiszuordnung}\\
Für $i \in \mathbb{N}$\\
$d(0) = d(1) = d(X_i) = 0$\\
\textbf{2. Induktionsregeln}\\
$d(\neg\psi) = 1 + d(\psi)$\\
Für $\ast \in \{\wedge, \vee, \rightarrow,\leftrightarrow \}$\\
$d((\varphi \ast \psi)) = 1 + max(d(\varphi), d(\psi))$

\section*{1.5 Modellierung}

\subsection*{(a)}
Definiere die Formel $\varphi$ so, dass jedem Objekt genau eine Menge zugeordnet ist:\\

$[\varphi]^{\beta} = 
\begin{cases}
1, & {\textrm{falls jedem Objekt genau eine Menge zugeordnet ist}}\\
0, & {\textrm{sonst}}
\end{cases}
$

\textbf{Idee:} Nutze eine DNF, so dass in jeder Klausel immer nur eine Variable nicht negiert wird. Somit ist die Formel nur dann wahr, wenn genau eine der Klauseln wahr wird.

$\varphi = \bigwedge_{i \in \{1..n\}}(\bigvee_{j \in \{1..m\}}(X_{i,j} \wedge \bigwedge_{k \in \{1..m\} \backslash \neq \{j\}} \neg X_{i,k}))$

\subsection*{(b)}
Definiere die Formel $\psi$ so, dass jeder Menge höchstens ein Element zugeordnet wird:\\

$[\psi]^{\beta} = 
\begin{cases}
1, & {\textrm{falls in jeder Menge höchstens ein Element zugeordnet ist}}\\
0, & {\textrm{sonst}}
\end{cases}
$

$\psi = \bigwedge_{j \in \{1..m\}}(
(\bigwedge_{j \in \{1..m\}} \neg X_{i,j}) \vee
\bigvee_{i \in \{1..n\}}(X_{i,j} \wedge \bigwedge_{k \in \{1..n\} \backslash \{i\}} \neg X_{k,j})
)$

\subsection*{(c)}

Für $[(\varphi \wedge \psi)]^{\beta}$ gilt

$
\begin{cases}
n \leq m, & {\textrm{Es gibt Belegungen für welche die Interpretation der Formel wahr ist}}\\
n > m, & {\textrm{Es gibt keine Belegung, für welche die Interpretation der Formel wahr ist}}
\end{cases}
$

\end{document}

