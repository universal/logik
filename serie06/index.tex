\documentclass[12pt,a4paper]{scrartcl}
\usepackage{ifpdf}
\usepackage[utf8]{inputenc}
\usepackage[ngerman]{babel}
\usepackage{amsmath}
\usepackage{amssymb}
\usepackage{fancyhdr}
\usepackage{listings}
\usepackage{color}
\usepackage{pdfpages}
\usepackage{stmaryrd}

\pagestyle{fancy}
\fancyhead{}
\fancyfoot{}
\fancyhead[LO,LE]{Johannes Hedtrich, Christopher Hlubek}
\fancyhead[RO,RE]{Serie 06}
\fancyfoot[CO,CE]{\thepage}

\title{Serie 06}
\author{Johannes Hedtrich, Christopher Hlubek}
\date{\today}
\begin{document}

\section*{6.3}
\noindent
\textbf{Voraussetzung:}\\
Seien $\Phi \subseteq F[\neg, \rightarrow]_{AL}$ und $\varphi \in F[\neg, \rightarrow]_{AL}$

\noindent
\textbf{Behauptung:}\\ 
$\text{Wenn } \Phi \vdash_H \varphi \text{ gilt, dann gilt }\Phi \models \varphi$.

\noindent
\textbf{Beweis:}\\
Es gelte $\Phi \vdash_H \varphi$.

\noindent
Somit gilt die Behauptung.

\section*{6.4}

\textbf{Voraussetzung:}\\
Sei $V_0 = \{1,3,5,7,9\}$ und $V_1 = \{2,4,6,8,10\}$, sei $s = 1$ und $t = 10$.

\textbf{Behauptung:}\\ 
Es existiert keine/eine Gewinnstrategie für Spieler Null.
\noindent
\textbf{Beweis: }

\end{document}
