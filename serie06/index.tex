\documentclass[12pt,a4paper]{scrartcl}
\usepackage{ifpdf}
\usepackage[utf8]{inputenc}
\usepackage[ngerman]{babel}
\usepackage{amsmath}
\usepackage{amssymb}
\usepackage{fancyhdr}
\usepackage{listings}
\usepackage{color}
\usepackage{pdfpages}
\usepackage{stmaryrd}

\pagestyle{fancy}
\fancyhead{}
\fancyfoot{}
\fancyhead[LO,LE]{Johannes Hedtrich, Christopher Hlubek}
\fancyhead[RO,RE]{Serie 06}
\fancyfoot[CO,CE]{\thepage}

\title{Serie 06}
\author{Johannes Hedtrich, Christopher Hlubek}
\date{\today}
\begin{document}

	\section*{6.3}
	\noindent
	\textbf{Voraussetzung:}\\
	Seien $\Phi \subseteq F[\neg, \rightarrow]_{AL}$ und $\varphi \in F[\neg, \rightarrow]_{AL}$

	\noindent
	\textbf{Behauptung:}\\ 
	Wenn $\Phi \vdash_H \varphi$ gilt, dann gilt $\Phi \models \varphi$.

	\noindent
	\textbf{Beweis:}\\
	Es gelte $\Phi \vdash_H \varphi$. Es gibt also eine Folge von Formeln $\varphi_0, \cdots , \varphi_{n-1} \in F[\neg, \rightarrow]_{AL}$ für die jeweils eine der drei Bedingungen "Voraussetzung", "Modus Ponens" oder "Axiom" gilt.

	Beweis per Induktion:
	Induktionsanfang "n=1":
	Es gibt nur eine Formel $\varphi_0 = \varphi$ im Hilbert-Beweis, die entweder eine Voraussetzung ist, d.h. $\varphi_0 \in \Phi$, oder durch ein Axiom gebildet wurde.
	1. Fall Voraussetzung:
	$\varphi \in \Phi$, also gilt offensichtlich $\Phi \models \varphi$.
	2. Fall Axiom
	Nach Lemma 2.17 gilt $\models \varphi$. Also gilt auch $\Phi \models \varphi$.

	Induktionsschritt:
	Es gelte $\Phi \models \varphi_{n-2}$ nach Induktionsvoraussetzung. Betrachte die letzte Formel $\varphi_{n-1}$, die entweder durch Voraussetzung, Modus Ponens oder ein Axiom gebildet wurde.
	1. Fall Voraussetzung:
	Dann gilt $\varphi_{n-1} \in \Psi$, daraus folgt $\Phi \models \varphi_{n-1}$.
	2. Fall Modus Ponens:
	Es gibt $j,k < n-1$ derart, dass $\varphi_k = \varphi_j \rightarrow \varphi_{n-1}$ gilt. Es gilt nach Induktionsvoraussetzung, dass $\Phi \models \varphi_j$ und $\Phi \models \varphi_k$. Nach Lemma 2.18 gilt ${\varphi_j, \varphi_k} \models \varphi_{n-1}$ und somit auch $\Phi \models \varphi_{n-1}$.
	3. Fall Axiom:
	Nach Lemma 2.17 gilt $\models \varphi_{n-1}$, also gilt auch $\Phi \models \varphi_{n-1}$.

	\noindent
	Somit gilt die Behauptung.

\section*{6.4}

\textbf{Voraussetzung:}\\
Sei $V_0 = \{1,3,5,7,9\}$ und $V_1 = \{2,4,6,8,10\}$, sei $s = 1$ und $t = 10$.

\textbf{Behauptung:}\\ 
Es existiert keine/eine Gewinnstrategie für Spieler Null.
\noindent
\textbf{Beweis: }

\end{document}
