\documentclass[12pt,a4paper]{scrartcl}
\usepackage{ifpdf}
\usepackage[utf8]{inputenc}
\usepackage[ngerman]{babel}
\usepackage{amsmath}
\usepackage{amssymb}
\usepackage{fancyhdr}
\usepackage{listings}
\usepackage{color}
\usepackage{pdfpages}
\usepackage{stmaryrd}
\usepackage{listings}
\lstset{
language=Ruby,%
numbers=left,%
numberstyle=\footnotesize,%
stepnumber=1,%
numbersep=5pt,%
backgroundcolor=\color{white},%
showspaces=false,%
showstringspaces=false,%
showtabs=false,%
frame=single,%
tabsize=2,%
captionpos=b,%
breaklines=true,%
breakatwhitespace=false,%
morecomment=[l]{//}
}


\author{Johannes Hedtrich}
\date{\today}
\begin{document}
  \section{Aussagenlogik}
  \begin{description}
    \item[$F_{AL}$] $0,1,X_i$ atomare Formeln\\
            $\varphi, \psi \in F_{AL}$: $\neg \varphi$, $\varphi \wedge \psi$, $\varphi \vee \psi$, $\varphi \rightarrow \psi$, $\varphi \leftrightarrow \psi \in F_{AL}$
    \item[$\beta: V_{AL} \dashrightarrow {0,1}$] $\beta$ passt zu $\varphi \in F_{AL}$, falls $vars(\varphi) \subseteq dom(\beta)$\\
            $\llbracket 0 \rrbracket^{\beta} = 0, \llbracket 1 \rrbracket^{\beta} = 1, \llbracket X_i \rrbracket^{\beta} = \beta(X_i)$\\
            $\varphi, \psi \in F_{AL} \text{und} * \in {\vee, \wedge, \rightarrow, \leftrightarrow} $: $\llbracket \neg \varphi \rrbracket^{\beta} = \overset{.}{\neg} \varphi, \llbracket(\varphi * \psi)\rrbracket^{\beta} = \llbracket \varphi \rrbracket^{\beta} \overset{.}{*} \llbracket \psi \rrbracket^{\beta}$
    \item[keine Formel ist echtes Anfangsstück einer anderen] Sind $\varphi, \psi \in F_{AL}$, so gilt $\varphi \not \sqsubset \psi$
    \item[Eindeutige Konstruktion] blub
    \item[$\varphi \equiv \psi$] für alle Belegungen $\beta$, die zu $\varphi, \psi$ passen gilt: $\llbracket \varphi \rrbracket^{\beta} = \llbracket \psi \rrbracket^{\beta}$
    \item[Koinzidenzlemma] Sei $\varphi \in F_{AL}$ und seien $\beta, \beta'$ Variablenbelegungen, die zu $\varphi$ passen. Falls $\beta|_{vars(\varphi)} = \beta'|_{vars(\varphi)}$, so $\llbracket \varphi \rrbracket^{\beta} = \llbracket \varphi \rrbracket^{\beta'}$
    \item[Aussagenlogische Gesetze] blub
    \item[Substitution $\sigma$] $\sigma: V_{AL} \dashrightarrow F_{AL}$
  \end{description}
\end{document}
