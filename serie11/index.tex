\documentclass[12pt,a4paper]{scrartcl}
\usepackage{ifpdf}
\usepackage[utf8]{inputenc}
\usepackage[ngerman]{babel}
\usepackage{amsmath}
\usepackage{amssymb}
\usepackage{fancyhdr}
\usepackage{listings}
\usepackage{color}
\usepackage{pdfpages}
\usepackage{stmaryrd}
\usepackage{listings}
\lstset{
language=Ruby,%
numbers=left,%
numberstyle=\footnotesize,%
stepnumber=1,%
numbersep=5pt,%
backgroundcolor=\color{white},%
showspaces=false,%
showstringspaces=false,%
showtabs=false,%
frame=single,%
tabsize=2,%
captionpos=b,%
breaklines=true,%
breakatwhitespace=false,%
morecomment=[l]{//}
}

\pagestyle{fancy}
\fancyhead{}
\fancyfoot{}
\fancyhead[LO]{Johannes Hedtrich, Christopher Hlubek}
\fancyhead[RO]{Serie 11}
\fancyfoot[CO]{\thepage}

\title{Serie 11}
\author{Johannes Hedtrich, Christopher Hlubek}
\date{\today}
\begin{document}
  \section*{11.3}
  
   \begin{align*}
      & (a) \text{erfüllbar} & f = g \\
      & (b) \text{erfüllbar} & 1-\text{elementige Trägermenge} \\
      & (c) \text{allgemeingültig} & "x_1 = x_2" \\
      & (d) \text{unerfüllbar} & ...\\
  \end{align*}
  
  \section*{11.4}
  \subsection*{Vor:} Sei $\Sigma$ eine Signatur und $\varphi \in F_{PL}(\Sigma)$.
  \subsection*{Beh:} $\forall x_0 \forall x_1 \varphi \equiv \forall x_1 \forall x_0 \varphi$
  \subsection*{Bew:}
  
  \begin{align*}
  &\text{Sei }I = (S, \beta)\text{eine} \Sigma-\text{Interpretation.}\\
  &\text{Es gilt }I \models \forall x_0 \forall x_1 \varphi \text{ gdw. } S,\beta [a/x_0][b/x_1] \models \varphi \text{ f.a. } a, b \in S \text{ gilt.}\\
  &\text{Da offensichtlich } \beta[a/x_0][b/x_1] = \beta[b/x_1][a/x_0] \text{, gilt auch}\\
  &S,\beta[b/x_1][a/x_0] \models \varphi \text{ f.a. } a, b \in S \text{ gdw. } I \models \forall x_1 \forall x_0 \varphi.\\
  \end{align*}

  \section*{11.5}
  
  $\varphi_{F_1}^{(i)} = P(f_c(f_a(e)), f_b(f_a(e))) \wedge
  P(f_a(f_b(e)), f_c(f_b(e))) \wedge
  P(f_b(f_b(f_a(e))), f_c(f_a(e)))$
  
  \begin{align*}
    \varphi_{F_1}^{(ii)} =& \forall x_0 \forall x_1 (P(x_0, x_1) \rightarrow
    P(f_c(f_a(x_0)), f_b(f_a(x_1)))\\ &\wedge
    P(f_a(f_b(x_0)), f_c(f_b(x_1))) \wedge
    P(f_b(f_b(f_a(x_0))), f_c(f_a(x_1)))
  \end{align*}
  $\varphi_{F_1} = \varphi_{F_1}^{(i)} \wedge \varphi_{F_1}^{(ii)} \rightarrow
  \exists x_0 P(x_0, x_0)$
  
  Die PCP-Instanz $F_1$ hat keine Lösung, da kein Teil eines Paares Präfix des anderen ist. 
  
  Die PCP-Instanz $F_2$ hat eine Lösung mit der Folge $P_2, P_1, P_1, P_1$.
  
\end{document}
