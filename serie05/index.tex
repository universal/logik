\documentclass[12pt,a4paper]{scrartcl}
\usepackage{ifpdf}
\usepackage[utf8]{inputenc}
\usepackage[ngerman]{babel}
\usepackage{amsmath}
\usepackage{amssymb}
\usepackage{fancyhdr}
\usepackage{listings}
\usepackage{color}
\usepackage{pdfpages}
\usepackage{stmaryrd}

\pagestyle{fancy}
\fancyhead{}
\fancyfoot{}
\fancyhead[LO,LE]{Johannes Hedtrich, Christopher Hlubek}
\fancyhead[RO,RE]{Serie 05}
\fancyfoot[CO,CE]{\thepage}

\title{Serie 05}
\author{Johannes Hedtrich, Christopher Hlubek}
\date{\today}
\begin{document}

\section*{5.3}

\textbf{Voraussetzung:}\\ 
Seien $\Phi \subseteq F_{AL}$ und $\varphi, \psi \in F_{AL}$

\noindent
\textbf{Behauptung:}\\ 

$\Phi \models \varphi$ und $\Phi \models \psi$ gdw. $\Phi \models \varphi \wedge \psi$

\noindent
\textbf{Beweis:}\\
"$\Rightarrow$"

Es gelte $\Phi \models \varphi$ und $\Phi \models \psi$. Sei $\beta$ eine Belegung die zu
$\Phi$, $\varphi$ und $\psi$ passt und gelte $\beta \models \Phi$. Dann gilt auch
$\beta \models \varphi$ und $\beta \models \psi$ nach Annahme. Nach Definition der
Erfüllbarkeit gilt $\llbracket \varphi \rrbracket^\beta = 1$ und
$\llbracket \psi \rrbracket^\beta = 1$, somit auch
$\llbracket \varphi \wedge \psi \rrbracket^\beta = 1$. Daraus folgt wieder
$\beta \models \varphi \wedge \psi$ und deshalb auch $\Phi \models \varphi \wedge \psi$,
da $\beta \models \Phi$ angenommen.
\\

\noindent
"$\Leftarrow$"

Es gelte $\Phi \models \varphi \wedge \psi$. Sei $\beta$ eine Belegung die zu
$\Phi$, $\varphi$ und $\psi$ passt und gelte $\beta \models \Phi$. Dann gilt auch
$\beta \models \varphi \wedge \psi$ nach Annahme. Nach Definition gilt
dann $\llbracket \varphi \wedge \psi \rrbracket^\beta = 1$. Somit gilt auch
$\llbracket \varphi \rrbracket^\beta = 1$ und
$\llbracket \psi \rrbracket^\beta = 1$. Das heißt $\beta \models \varphi$ und $\beta \models \psi$.
Daraus folgt nach Annahme von $\beta \models \Phi$, dass
$\Phi \models \varphi$ und $\Phi \models \varphi$.

\noindent
Somit gilt die Behauptung.

\section*{5.4}

\textbf{Behauptung:}\\ Die Formel $\varphi = \bigwedge \bigvee\{\{X_1, X_2, \neg X_3\},\{X_3, \neg X_4, X_5\},\{\neg X_1, X_2, \neg X_4\}, \{\neg X_2\},\{\neg X_5\},\{X_2, X_4\}\}$ unerfüllbar.

\noindent
\textbf{Beweis: }

\begin{align}
  & \{X_1, X_2, \neg X_3\} & (V)\\
  & \{X_3, \neg X_4\, X_5\} & (V)\\
  & \{X_1, X_2, \neg X_4\, X_5\} & (R) & auf 1 + 2\\
  & \{\neg X_1, X_2, \neg X_4\} & (V)\\
  & \{X_2, \neg X_4, X_5\} & (R) & auf 3 + 4\\
  & \{X_2, X_4\} & (V)\\
  & \{X_2, X_5\} & (R) & auf 5 + 6\\
  & \{\neg X_2\} & (V)\\
  & \{X_5\} & (R) & auf 7 + 8\\
  & \{\neg X_5\} & (V)\\
  & \{\} & (R) & auf 9 + 10
\end{align}

Somit ist $\varphi$ unerfüllbar.

\end{document}
