\documentclass[12pt,a4paper]{scrartcl}
\usepackage{ifpdf}
\usepackage[utf8]{inputenc}
\usepackage[ngerman]{babel}
\usepackage{amsmath}
\usepackage{amssymb}
\usepackage{fancyhdr}
\usepackage{listings}
\usepackage{color}
\usepackage{pdfpages}
\usepackage{stmaryrd}

\pagestyle{fancy}
\fancyhead{}
\fancyfoot{}
\fancyhead[LO,LE]{Johannes Hedtrich, Christopher Hlubek}
\fancyhead[RO,RE]{Serie 05}
\fancyfoot[CO,CE]{\thepage}

\title{Serie 05}
\author{Johannes Hedtrich, Christopher Hlubek}
\date{\today}
\begin{document}

\section*{5.3}

\textbf{Voraussetzung:}\\ 
Seien $\phi \subseteq F_{AL}$ und $\varphi, \psi \in F_{AL}$

\noindent
\textbf{Behauptung:}\\ 

$\psi \models \varphi$ und $\psi \models \psi \Leftrightarrow \psi \models \varphi \wedge \psi$

\noindent
\textbf{Beweis:}\\
"$\Rightarrow$"

Es gelte $\psi \models \varphi$ und $\psi \models \psi$. Sei nun $\beta$ eine Belegung die zu $\psi$ und ...

\noindent
"$\Leftarrow$"

Es gelte $\psi \models \varphi \wedge \psi$.



\section*{5.4}

\textbf{Behauptung:}\\ Die Formel $\varphi = \bigwedge \bigvee\{\{X_1, X_2, \neg X_3\},\{X_3, \neg X_4, X_5\},\{\neg X_1, X_2, \neg X_4\}, \{\neg X_2\},\{\neg X_5\},\{X_2, X_4\}\}$ unerfüllbar.

\noindent
\textbf{Beweis: }

\begin{align}
  & \{X_1, X_2, \neg X_3\} & (V)\\
  & \{X_3, \neg X_4\, X_5\} & (V)\\
  & \{X_1, X_2, \neg X_4\, X_5\} & (R) & auf 1 + 2\\
  & \{\neg X_1, X_2, \neg X_4\} & (V)\\
  & \{X_2, \neg X_4, X_5\} & (R) & auf 3 + 4\\
  & \{X_2, X_4\} & (V)\\
  & \{X_2, X_5\} & (R) & auf 5 + 6\\
  & \{\neg X_2\} & (V)\\
  & \{X_5\} & (R) & auf 7 + 8\\
  & \{\neg X_5\} & (V)\\
  & \{\} & (R) & auf 9 + 10
\end{align}

hier fehlt noch bisschen blabla

Somit ist $\varphi$ unerfüllbar.

\end{document}
