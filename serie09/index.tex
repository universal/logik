\documentclass[12pt,a4paper]{scrartcl}
\usepackage{ifpdf}
\usepackage[utf8]{inputenc}
\usepackage[ngerman]{babel}
\usepackage{amsmath}
\usepackage{amssymb}
\usepackage{fancyhdr}
\usepackage{listings}
\usepackage{color}
\usepackage{pdfpages}
\usepackage{stmaryrd}
\usepackage{listings}
\lstset{
language=Ruby,%
numbers=left,%
numberstyle=\footnotesize,%
stepnumber=1,%
numbersep=5pt,%
backgroundcolor=\color{white},%
showspaces=false,%
showstringspaces=false,%
showtabs=false,%
frame=single,%
tabsize=2,%
captionpos=b,%
breaklines=true,%
breakatwhitespace=false,%
morecomment=[l]{//}
}


\pagestyle{fancy}
\fancyhead{}
\fancyfoot{}
\fancyhead[LO]{Johannes Hedtrich, Christopher Hlubek}
\fancyhead[RO]{Serie 09}
\fancyfoot[CO]{\thepage}

\title{Serie 09}
\author{Johannes Hedtrich, Christopher Hlubek}
\date{\today}
\begin{document}

  \section*{9.3}
  \noindent
  \textbf{Voraussetzung:}\\
  
  Seien $\Sigma$ eine Signatur, $t$ ein $\Sigma$-T, $\textbf{S}$ eine $\Sigma$-Struktur und $\beta$, $\beta'$ zu $t$ passende $\textbf{S}$-Belegungen.

  \noindent
  \textbf{Behauptung:}\\ 
  
  Ist $\beta|_{vars(t)} = \beta'|_{vars(t)}$, so gilt $[t]^{\beta}_{\textbf{S}} = [t]^{\beta'}_{\textbf{S}}$

  \noindent
  \textbf{Beweis:}\\
  
  Beweis durch vollständige Induktion über den Termaufbau.\\
  \\
  Induktionsanfang:\\
  Ist $t = x_i$, so gilt:
  
  \begin{align*}
  \llbracket t \rrbracket^{\beta}_{\textbf{S}} &= \llbracket x_i \rrbracket^{\beta}_{\textbf{S}}\\
                                               &= \beta(x_i) &\text{nach Def. der Interpretation}\\
                                               &= \beta'(x_i) &\text{da $x_i \in \text{vars}(t)$}\\
                                               &= \llbracket x_i \rrbracket^{\beta'}_{\textbf{S}} &\text{Def. Interpretation}\\
                                               &= \llbracket t \rrbracket^{\beta'}_{\textbf{S}}
  \end{align*}
  
  
  Ist $t = c$, mit $c \in \Sigma$, so gilt:
  
  \begin{align*}  
  \llbracket t \rrbracket^{\beta}_{\textbf{S}} &= \llbracket c \rrbracket^{\beta}_{\textbf{S}}\\
                                                &= c^{\textbf{S}} & \text{nach Def. der Interpretation}\\
                                                &= \llbracket c \rrbracket^{\beta'}_{\textbf{S}}\\
                                                &= \llbracket t \rrbracket^{\beta'}_{\textbf{S}}
  \end{align*}
  
  Induktionsschluss:\\
  Ist $t = f(t_0, ...,t_{n-1})$ mit $f // n \in \Sigma$ und sind $t_0, ..., t_{n-1}$ $\Sigma$-Terme, so gilt:\\
  \begin{align*}
  \llbracket t \rrbracket^{\beta}_{\textbf{S}} &= \llbracket f(t_0, ...,t_{n-1}) \rrbracket^{\beta}_{\textbf{S}}\\
                                               &= f^{\textbf{S}}(\llbracket t_0 \rrbracket^{\beta}_{\textbf{S}}, ...,\llbracket t_{n-1} \rrbracket^{\beta}_{\textbf{S}} &\text{Def. Interpretation}\\
                                               &= f^{\textbf{S}}(\llbracket t_0 \rrbracket^{\beta'}_{\textbf{S}}, ...,\llbracket t_{n-1} \rrbracket^{\beta'}_{\textbf{S}}) &\text{nach IV}\\
                                               &= \llbracket f(t_0, ...,t_{n-1}) \rrbracket^{\beta'}_{\textbf{S}} &\text{Def. Interpretation}\\
                                               &= \llbracket t\rrbracket^{\beta'}_{\textbf{S}}
  \end{align*}  
  \section*{9.4}
  \noindent
  \textbf{(a)}\\
  
  asd\\
  \textbf{(b)}\\
\end{document}
